\subsection*{Lemuří tábor}
\label{sub:lemurui_tabor}

Letos jsme my, Lemuři, v čase posledního týdne tábora podnikali své lemurské pobývání. Tábořiště bylo situováno na polootevřené louce nedaleko Mirotic u Čimelic na malebném meandrujícím potoku Lomnice. Bylo to výjimečné místo. Otevřenou polovinou louky se na nás smálo velké široké údolí s alejí mohutných topolů. Netrpěli jsme zde nedostatkem vody, netrpěli jsme…netrpěli jsme ničím. Snad jen vedro bylo sem tam odzbrojující. Vždy nám ale byl po ruce stín, tam nám bylo dobře. Krom toho skoro každý den byl plný aktivit mimotábořištních. Jednou jsme šli na výlet, jindy byl náš cíl blízký rybník, pak byla neděle a my šli do kostela.

Minulý rok Lemuři skládali roverský slib, a tím vědomě vstupovali do roverského a z dětského času vycházeli. Roverské heslo „Sloužím!“ by se mělo nést každou roverskou akcí, a tak i my byli sloužit. Byli jsme v domově seniorů s paměťovými chorobami, otloukávali vlhkou zeď na tamější faře, vyměňovali brzdové destičky u feldy.  

Náš tábor byl ohromně naplněn pohodou a k té přispívalo hned několik věcí.

Jídlo. Hodně epesňovelkolepých jídel denně nám moc vyhovovalo. Ani čas strávený nad jejich uvařením nám nebral ani trochu chutě. Ba naopak. Měli jsme ovocné knedlíky (ty trvaly obzvlášť dlouho), rýži s moc dobrým masem, kokosem, jablky, zelím… fotky vám poví vše co potřebujete vědět.   

Dalším aspektem byly každovečerní pohodičky u ohně, v sauně (tu jsme měli dvakrát a byla moooc dobrá :-) ), s kytarou i bez ní. Přes den nám mohli být útočištěm i hamaky, ze kterých jsme si vyrobili pětipatrovou palandu.  Básně a legendy se budou na toto psát po dlouhé dekády. 

\podpis{Wexlák}